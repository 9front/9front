\documentclass{howto}
\usepackage{distutils}
% $Id: whatsnewXY.tex 33728 2003-07-31 01:17:22Z montanaro $

% When creating a new ``What's New'' document, copy this to
% ../whatsnew/whatsnewXY.tex, where X is replaced by the major version
% number and Y, by the minor version number for the release of Python
% being described.
%
% The following replacements need to be made in the text:
%
% X.Y   -- the version of Python this document describes
% X.Y-1 -- previous minor release (not a maintenance release)
% X.Y-2 -- minor release before that one (optional; search the
%          template to see the usage
%
% Once done, write and edit to your heart's content!

\title{What's New in Python X.Y}
\release{0.0}
\author{Young Author}
\authoraddress{\email{ya@example.com}}

\begin{document}
\maketitle
\tableofcontents

This article explains the new features in Python X.Y.  No release date
for Python X.Y has been set; expect that this will happen next year.

% Compare with previous release in 2 - 3 sentences here.

This article doesn't attempt to provide a complete specification of
the new features, but instead provides a convenient overview.  For
full details, you should refer to the documentation for Python X.Y.
% add hyperlink when the documentation becomes available online.
If you want to understand the complete implementation and design
rationale, refer to the PEP for a particular new feature.


%======================================================================

% Large, PEP-level features and changes should be described here.


%======================================================================
\section{Other Language Changes}

Here are all of the changes that Python X.Y makes to the core Python
language.

\begin{itemize}
\item TBD

\end{itemize}


%======================================================================
\subsection{Optimizations}

\begin{itemize}

\item Optimizations should be described here.

\end{itemize}

The net result of the X.Y optimizations is that Python X.Y runs the
pystone benchmark around XX\% faster than Python X.Y-1.%
% only use the next line if you want to do the extra work ;) :
% and YY\% faster than Python X.Y-2.


%======================================================================
\section{New, Improved, and Deprecated Modules}

As usual, Python's standard library received a number of enhancements and
bug fixes.  Here's a partial list of the most notable changes, sorted
alphabetically by module name. Consult the
\file{Misc/NEWS} file in the source tree for a more
complete list of changes, or look through the CVS logs for all the
details.

\begin{itemize}

\item Descriptions go here.

\end{itemize}


%======================================================================
% whole new modules get described in \subsections here


% ======================================================================
\section{Build and C API Changes}

Changes to Python's build process and to the C API include:

\begin{itemize}

\item Detailed changes are listed here.

\end{itemize}


%======================================================================
\subsection{Port-Specific Changes}

Platform-specific changes go here.


%======================================================================
\section{Other Changes and Fixes \label{section-other}}

As usual, there were a bunch of other improvements and bugfixes
scattered throughout the source tree.  A search through the CVS change
logs finds there were XXX patches applied and YYY bugs fixed between
Python X.Y-1 and X.Y.  Both figures are likely to be underestimates.

Some of the more notable changes are:

\begin{itemize}

\item Details go here.

\end{itemize}


%======================================================================
\section{Porting to Python X.Y}

This section lists previously described changes that may require
changes to your code:

\begin{itemize}

\item Everything is all in the details!

\end{itemize}


%======================================================================
\section{Acknowledgements \label{acks}}

The author would like to thank the following people for offering
suggestions, corrections and assistance with various drafts of this
article: .

\end{document}
